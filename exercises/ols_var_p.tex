\section[Ordinary Least Squares Estimation of VAR{(p)}]{Ordinary Least Squares Estimation of VAR{(p)}\label{ex:OrdinaryLeastSquaresEstimationVARp}}
Consider the VAR{(p)} model with a constant written in the compact form
\begin{align*}
y_t = [c, A_1, \ldots, A_p] Z_{t-1} + u_t = A Z_{t-1}+ u_t
\end{align*}
where \(Z_{t-1}=(1,y_{t-1}',\ldots,y_{t-p}')'\)
and \(u_t\) is assumed to be iid white noise with non-singular covariance matrix \(\Sigma_u\).
Given a sample of size \(T\), \(y_1, \ldots, y_T\), and \(p\) presample vectors, \(y_{-p+1},\ldots,y_{0}\),
  ordinary least squares for each equation separately results in efficient estimators.
The OLS estimator is
\begin{align*}
\hat{A} = \left[ \hat{c}, \hat{A_1}, \ldots ,\hat{A_p}\right] = \left(\sum_{t=1}^{T}y_t Z_{t-1}'\right){\left(\sum_{t=1}^{T}Z_{t-1}Z_{t-1}'\right)}^{-1} = Y Z'{(ZZ')}^{-1}
\end{align*}
where \(Y=[y_1, \ldots, y_T]\) and \(Z=[Z_0, \ldots, Z_{T-1}]\).
More precisely, stacking the columns of \(A = [c, A_1,\ldots ,A_p]\) in the vector \(\alpha = vec(A)\),
\begin{align*}
\sqrt{T}\left(\hat{\alpha} - \alpha \right) \overset{d}{\rightarrow} \mathcal{N}(0, \Sigma_{\hat{\alpha}})
\end{align*}
where \(\Sigma_{\hat{\alpha}} = plim(\frac{1}{T}ZZ')^{-1}\otimes \Sigma_u\), if the process is stable.
Under fairly general assumptions this estimator has an asymptotic normal distribution.
A sufficient condition for the consistency and asymptotic normality of \(\hat{A}\) would be
  that \(u_t\) is a continuous white noise random variable with four finite moments.
A consistent estimator of the innovation covariance matrix \(\Sigma_u\) is, for example,
\begin{align*}
\hat{\Sigma}_u = \frac{\hat{U}\hat{U}'}{T-Kp-1}
\end{align*}
where \(\hat{U} = Y - \hat{A}Z\) are the OLS residuals.
Thus, in large samples,
\begin{align*}
vec(\hat{A}) \overset{a}{\sim}\mathcal{N}(vec(A),{(ZZ')}^{-1}\otimes \hat{\Sigma}_u)
\end{align*}
where \(\overset{a}{\sim}\) denotes the approximate large-sample distribution.
In other words, asymptotically the usual t-statistics can be used for testing restrictions on individual coefficients
  and for setting up confidence intervals.
\begin{enumerate}
\item What are the dimensions of \(y_t\), \(Y\), \(u_t\), \(U\), \(c\), \(A_1\), \ldots, \(A_p\), \(A\), \(\alpha \), \(Z_{t-1}\), \(Z\), \(\Sigma_u\) and \(\Sigma_{\hat{\alpha}}\).
\item Modify your \texttt{ARpOLS} function such that it is able to estimate VAR{(p)} models.
  Save the modified function as \texttt{VARReducedForm}.
\item Consider data given in \texttt{threeVariableVAR.csv} for \(y_t = (\Delta gnp_t,i_t,\Delta p_t)'\),
  where \(gnp_t\) denotes the log of U.S. real GNP,
  \(p_t\) the corresponding GNP deflator in logs,
  and \(i_t\) the federal funds rate, averaged by quarter.
The estimation period is restricted to 1954q4 to 2007q4. 
\begin{itemize}
	\item Load the data and visualize it. Comment whether you think the data looks stationary.
	\item Estimate a VAR{(4)} model using the \texttt{VARReducedForm} function.
	Examine the stability of the estimated process
	  and the significance of the estimated parameters at a \(95\% \) level.
\end{itemize}
\end{enumerate}

\paragraph{Readings}
\begin{itemize}
\item \textcite[Ch.~2.3]{Kilian.Lutkepohl_2017_StructuralVectorAutoregressive}
\end{itemize}

\begin{solution}\textbf{Solution to \nameref{ex:OrdinaryLeastSquaresEstimationVARp}}
\ifDisplaySolutions
\begin{enumerate}
\item The dimensions are: \(y_t\) is \(K \times 1\), \(Y\) is \(K \times T\), \(u_t\) is \(K \times 1\),
\(U\) is \(K \times T\), \(c\) is \(K \times 1\), \(A_1\) is \(K \times K\), \ldots, \(A_p\) is \(K \times K\),
\(A\) is \(K \times (1+pK)\), \(\alpha\) is \((pK^2+K) \times 1\),
\(Z_{t-1}\) is \((1+pK) \times 1\), \(Z\) is \((Kp+1) \times T\),
\(\Sigma_u\) is \(K \times K\) and \(\Sigma_{\hat{\alpha}}\) is \((pK^2+K) \times (pK^2+K)\).
\lstinputlisting[style=Matlab-editor,basicstyle=\mlttfamily,title=\lstname]{progs/matlab/VARpDimensionsIllustration.m}

\item A modified version looks like this.
Note that it also includes OLS estimation of each equation in turn.
\lstinputlisting[style=Matlab-editor,basicstyle=\mlttfamily,title=\lstname]{progs/matlab/VARReducedForm.m}	

\item The OLS estimation of the three variables VAR model might look like this:
\lstinputlisting[style=Matlab-editor,basicstyle=\mlttfamily,title=\lstname]{progs/matlab/threeVariableVAROLS.m}
The data for Federal Funds Rate as well as the GNP Deflator Inflation do seem to have some trend in it, but nothing serious.
\end{enumerate}

\fi
\newpage
\end{solution}