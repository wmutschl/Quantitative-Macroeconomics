\section[Central Limit Theorem For Dependent Data]{Central Limit Theorem For Dependent Data\label{ex:CentralLimitTheoremDependentData}}
Suppose that the sequence \(Y_{1},Y_{2},\ldots \) is an AR{(1)} process, i.e.\
\begin{align*}
Y_{t}-\mu =\phi \left(Y_{t-1}-\mu\right) +\varepsilon _{t}
\end{align*}
where \(\varepsilon _{t}\sim iid(0,\sigma _{\varepsilon }^{2})\) is
(not necessarily but in our case) normally distributed and \(|\phi |<1\).

\begin{enumerate}
\item
Briefly state and describe the intuition of the \enquote{Lindeberg-Levy Central Limit Theorem} for iid random variables.
What does \enquote{convergence in distribution} mean?
Why can we not use the theorem for the AR{(1)} process?

\item
Show that \(Y_t\) has mean equal to \(\mu \) and finite variance equal to \(\sigma_\varepsilon^2/(1-\phi^2)\).

\item To derive the asymptotic distribution of the sample mean, do the following steps:
\begin{enumerate}
  \item
  Derive the asymptotic distribution of \(\frac{1}{\sqrt{T} } \sum_{t=1}^T \varepsilon_t\).

  \item
   Show that
  \begin{align*}
  \frac{1}{\sqrt{T}} \sum_{t=1}^T \varepsilon_t = \sqrt{T}\left[(1-\phi)\left(\hat{\mu}-\mu\right) + \phi\left(\frac{Y_T - Y_0}{T}\right)\right]
  \end{align*}
  with \(\hat{\mu} =\frac{1}{T}\sum_{t=1}^{T}Y_{t}\).

  \item
  Show that
  \begin{align*}
  \textsl{plim}\left[\frac{\phi}{1-\phi}\left(\frac{Y_T - Y_0}{\sqrt{T}}\right)\right] = 0
  \end{align*}
  \\
  \emph{Hint: Use Tchebychev's Inequality,
  i.e.\ for a random variable \(X\) with expectation \(\mu_x\)
  and finite variance \(\sigma_x^2\):}
  \begin{align*}
  \Pr(|X-\mu_x|> \delta) \leq \frac{\sigma_x^2}{\delta^2}
  \end{align*}
  \emph{for any small real number \(\delta>0\).}

  \item
  Put your results of (a), (b) and (c) together and derive the asymptotic distribution of the sample mean.
  That is, show that
  \begin{align*}
  Z_{T} =\sqrt{T}\frac{\hat{\mu} -\mu }{\sigma_Z} \overset{d}{\rightarrow} U \sim N(0,1)
  \end{align*}
  for \(\sigma_Z=\sqrt{\sigma_\varepsilon^2/{(1-\phi)}^2}\).
\end{enumerate}

\item
 Write a program to demonstrate the central limit theorem for the AR{(1)} process. To this end:

\begin{itemize}

  \item
  Simulate \(B=5000\) stationary (e.g.\
  \(\phi=0.8\)) AR{(1)} processes with each \(T=10000\) observations.
  Store these in a \(T \times B\) matrix \(Y\).

  \item
  Compute \(\hat{\mu}\) for each column of \(Y\).

  \item
  Plot the histograms of the standardized variables according to the Lindeberg-Levy Central Limit Theorem:
  \begin{align*}
  \widetilde{Z}_T = \sqrt{T}\frac{\hat{\mu}-\mu}{\sigma_{\varepsilon }/\sqrt{1-\phi^2}}
  \end{align*}
  and of the correct standardized variables that we derived in 3(d):
  \begin{align*}
  Z_T = \sqrt{T}\frac{\hat{\mu}-\mu}{\sigma_{\varepsilon }/(1-\phi)}
  \end{align*}
  Compare the histograms to the standard normal distribution.

\end{itemize}

\end{enumerate}

\paragraph{Readings}
\begin{itemize}
\item \textcite{Crack.Ledoit_2010_CentralLimitTheorems}
\item \textcite[App. C]{Lutkepohl_2005_NewIntroductionMultiple}
\item \textcite[App. C]{Neusser_2016_TimeSeriesEconometrics}
\item \textcite[Ch. 5]{White_2001_AsymptoticTheoryEconometricians}
\end{itemize}

\begin{solution}\textbf{Solution to \nameref{ex:CentralLimitTheoremDependentData}}
\ifDisplaySolutions%
\begin{enumerate}

\item
We usually consider the \emph{Lindeberg-Levy Central Limit Theorem} for identically and independently (iid) random variables
  with finite mean \( \mu \) and finite variance \(\sigma_Y^2\).
Then the \emph{Lindeberg-Levy Central Limit Theorem} establishes the distribution of the sample mean \(\overline{Y}_T\) for a growing sample size:
\begin{align*}
\sqrt{T} (\hat{\mu}-\mu) \overset{d}{\rightarrow} N(0,\sigma_Y^2)
\end{align*}
with \(\hat{\mu} = \bar{Y}_T = \frac{1}{T} \sum_{t=1}^T y_t\).
Or more compactly using standardized variables:
\begin{align*}
z = \sqrt{T}\frac{\hat{\mu}-\mu}{\sigma_Y}\overset{d}{\rightarrow} U \sim N(0,1)
\end{align*}
The central limit theorem is closely related to the LLN,
  but while the LLN is a statement about \textbf{converging to a constant},
  central limit theorems look at \textbf{convergence in distribution},
  i.e.\ the distribution of the sample mean.
More formally, a sequence \(\bar{Y}_{1},\bar{Y}_{2},\ldots \) of random variables with distribution functions
\(F_{1},F_{2},\ldots \) converges \textbf{in distribution (weakly; in law)}
to a variable \(\mu \) with distribution function \(F\), if
\begin{equation*}
\lim_{T\rightarrow \infty }F_{T}(x)=F(x)
\end{equation*}
for all \(x\in \mathbb{R}\) where \(F(x)\) is continuous.
Typically, we use the following notation for this:
\begin{equation*}
\bar{Y}_{T}\overset{d}{\rightarrow }\mu
\end{equation*}

Unfortunately, the \emph{Lindeberg-Levy Central Limit Theorem} does not apply
  for the AR{(1)} process as we have dependent and not iid data.
For stationary AR{(1)} processes, we can however use similar central limit theorems
  either for Martingale-Difference-processes or mixing processes.

\item
First let's derive the expectation and variance of the AR{(1)} process with \(|\phi|<1\).
For this, we use recursive substitution techniques given a starting value \(Y_0\):
\begin{align*}
Y_t = (1-\phi)(1+\phi+\phi^2+\cdots+\phi^T)\mu + \varepsilon_t + \phi \varepsilon_{t-1} + \phi^2 \varepsilon_{t-2} + \cdots + \phi^T \varepsilon_{t-T} + \phi^{T+1} Y_0
\end{align*}
Note that \(\lim_{T\rightarrow \infty} \phi^{T+1} = 0\)
  and \(\lim_{T\rightarrow \infty} \sum_{j=0}^\infty \phi^j = \frac{1}{1-\phi}\), since \(|\phi|<1\).
The AR{(1)} process with \(|\phi|<1\) can therefore be equally represented by
\begin{align*}
Y_t = \mu + \sum_{j=1}^\infty \phi^j \varepsilon_{t-j}
\end{align*}
Its expectation and variance are then equal to
\begin{align*}
E[Y_t] &= \mu + \sum_{j=1}^\infty \phi^j E[\varepsilon_{t-j}] = \mu
\\
Var[Y_t] &= \sum_{j=1}^\infty {(\phi^j)}^2 Var[\varepsilon_{t-j}] = \sum_{j=1}^\infty {(\phi^2)}^j \sigma_\varepsilon^2 = \frac{\sigma_\varepsilon^2}{1-\phi^2}
\end{align*}
where we use the white noise property of \(\varepsilon_t\).

\item
Let's derive the asymptotic distribution of the sample mean:

\begin{enumerate}
  \item
  Due to the white noise assumption on \(\varepsilon_t\),
    we can use the Lindeberg-Levy central limit theorem such that
  \begin{equation*}
  \sqrt{T} \left(\frac{1}{T} \sum_{t=1}^T \varepsilon_t \right) = \frac{1}{\sqrt{T}} \sum_{t=1}^T \varepsilon_t  \overset{d}{\rightarrow} U_\varepsilon \sim N(0,\sigma_\varepsilon^2)
  \end{equation*}
  \item
  Let's have a look at \(\frac{1}{\sqrt{T}} \sum_{t=1}^T \varepsilon_t\):
  \begin{align*}
  \frac{1}{\sqrt{T}} \sum_{t=1}^T \varepsilon_t
  &= \frac{1}{\sqrt{T}} \sum_{t=1}^T \left[(Y_t-\mu)-\phi(Y_{t-1}-\mu)\right]
  \\
  &= \frac{1}{\sqrt{T}} \left[\sum_{t=1}^T (Y_t-\mu)- \phi\sum_{t=1}^T(Y_{\color{red}{t-1}}-\mu)\right]
  \\
  &= \frac{1}{\sqrt{T}} \left[\sum_{t=1}^T (Y_t-\mu)- \phi\left[\sum_{t=1}^T(Y_{\color{red}{t}}-\mu)-\color{red}{(Y_T - Y_0)}\right]\right]
  \\
  &= \sqrt{T} \left[\frac{1}{T}\sum_{t=1}^T (Y_t-\mu)- \phi\left[\frac{1}{T}\sum_{t=1}^T(Y_{t}-\mu)-\left(\frac{Y_T - Y_0}{T}\right)\right]\right]
  \\
  &= \sqrt{T} \left[\hat{\mu}-\mu- \phi\left[\hat{\mu}-\mu-\left(\frac{Y_T - Y_0}{T}\right)\right]\right]
  \\
  &= \sqrt{T}\left[(1-\phi)\left(\hat{\mu}-\mu\right) + \phi\left(\frac{Y_T - Y_0}{T}\right)\right]
  \end{align*}

  \item
  Using the definition of the probability limit, i.e.\
  \begin{align*}
  \textsl{plim}\left[\frac{\phi}{1-\phi}\left(\frac{Y_T - Y_0}{\sqrt{T}}\right)\right] = 0
  \end{align*}
  is shorthand notation for
  \begin{align*}
  \lim_{T\rightarrow \infty} \Pr\left(\left|\frac{\phi}{1-\phi}\left(\frac{Y_T - Y_0}{\sqrt{T}}\right)\right|> \delta\right) = 0
  \end{align*}
  We are going to show this by using Tchebychev's Inequality.
  That is, according to the inequality we have:
  \begin{align*}
  \Pr\left(\left|\frac{\phi}{1-\phi}\left(\frac{Y_T - Y_0}{\sqrt{T}}\right)\right|> \delta\right) \leq \frac{1}{\delta^2} var\left[\frac{\phi}{1-\phi}\left(\frac{Y_T - Y_0}{\sqrt{T}}\right)\right]
  \end{align*}
  for any \(\delta>0\).
  Let's have a look at \(var\left[\frac{\phi}{1-\phi}\left(\frac{Y_T - Y_0}{\sqrt{T}}\right)\right]\):
  \begin{multline*}
  var\left[\frac{\phi}{1-\phi}\left(\frac{Y_T - Y_0}{\sqrt{T}}\right)\right]
  = \frac{1}{T}{\left(\frac{\phi}{1-\phi}\right)}^2 Var[Y_T - Y_0]
  \\
  = \frac{1}{T}{\left(\frac{\phi}{1-\phi}\right)}^2 \left(Var[Y_T] + Var[Y_0] - 2 Cov[Y_T,Y_0]\right)
  \\
  = \frac{1}{T}{\left(\frac{\phi}{1-\phi}\right)}^2 \left[\frac{\sigma_\varepsilon^2}{1-\phi^2} + \frac{\sigma_\varepsilon^2}{1-\phi^2} - 2 Corr[Y_T,Y_0]\sqrt{\frac{\sigma_\varepsilon^2}{1-\phi^2}}\sqrt{\frac{\sigma_\varepsilon^2}{1-\phi^2}} \right]
  \\
  \leq \frac{1}{T}{\left(\frac{\phi}{1-\phi}\right)}^2 4 \left(\frac{\sigma_\varepsilon^2}{1-\phi^2}\right)
  \end{multline*}
  since \(corr(Y_T,Y_0) \geq -1\).

  Thus for any \(\delta>0\), we have
  \begin{align*}
  \Pr\left(\left|\frac{\phi}{1-\phi}\left(\frac{Y_T - Y_0}{\sqrt{T}}\right)\right|> \delta\right) \leq \frac{1}{\delta^2} \frac{1}{T}{\left(\frac{\phi}{1-\phi}\right)}^2 4 \left(\frac{\sigma_\varepsilon^2}{1-\phi^2}\right)
  \end{align*}
  In the limit for \(T\rightarrow \infty \) the right hand side converges to 0; hence:
  \begin{align*}
  \lim_{T\rightarrow \infty} \Pr\left(\left|\frac{\phi}{1-\phi}\left(\frac{Y_T - Y_0}{\sqrt{T}}\right)\right|> \delta\right) = 0.
  \end{align*}
  or using our shorthand notation:
  \begin{align*}
  \textsl{plim}\left[\frac{\phi}{1-\phi}\left(\frac{Y_T - Y_0}{\sqrt{T}}\right)\right] = 0
  \end{align*}
  
  \item
  Now, let's go back to
  \begin{align*}
  \frac{1}{\sqrt{T}} \sum_{t=1}^T \varepsilon_t = \sqrt{T}\left[(1-\phi)\left(\hat{\mu}-\mu\right) + \phi\left(\frac{Y_T - Y_0}{T}\right)\right]
  \end{align*}
  Let's divide by \((1-\phi)\)
  \begin{align*}
  \frac{\frac{1}{\sqrt{T}} \sum_{t=1}^T \varepsilon_t}{1-\phi} & = \sqrt{T}\left(\hat{\mu}-\mu\right) + \frac{\phi}{1-\phi}\left(\frac{Y_T - Y_0}{\sqrt{T}}\right)
  \end{align*}

  For the left-hand-side we have
  \begin{align*}
  \frac{\frac{1}{\sqrt{T}} \sum_{t=1}^T \varepsilon_t}{1-\phi} \overset{d}{\rightarrow} \tilde{U_\varepsilon} \sim N\left(0,\frac{\sigma_\varepsilon^2}{{(1-\phi)}^2}\right)
  \end{align*}

  This is also the distribution of the right-hand side.
  However, in the limit, the right-hand side actually simplifies
    as we just derived that \(\textsl{plim}\left[\frac{\phi}{1-\phi}\left(\frac{Y_T - Y_0}{\sqrt{T}}\right)\right] = 0\).
  Therefore:
  \begin{align*}
  \sqrt{T}\left(\hat{\mu}-\mu\right) \overset{d}{\rightarrow} \tilde{U} \sim N\left(0,\frac{\sigma_\varepsilon^2}{{(1-\phi)}^2}\right)
  \end{align*}
  and we're done.
  We have just derived the distribution to which the sample mean \(\hat{\mu} = \overline{Y}_T\) of an AR{(1)} process converges to asymptotically.
  Note that the required standardization is different than the Lindeberg-Levy central limit theorem would suggest.
  The correct standardized variable to consider is:
  \begin{align*}
  Z_T = \sqrt{T}\frac{\hat{\mu}-\mu}{\sigma_Z} \overset{d}{\rightarrow} U \sim N(0,1)
  \end{align*}
  where we need to set \(\sigma_Z^2 = \frac{\sigma_\varepsilon^2}{{(1-\phi)}^2}\) to get convergence to the standard normal distribution.

\end{enumerate}

\item A possible script:
\lstinputlisting[style=Matlab-editor,basicstyle=\mlttfamily,title=\lstname]{progs/matlab/CentralLimitDependentData.m}
\end{enumerate}
\fi
\newpage
\end{solution}