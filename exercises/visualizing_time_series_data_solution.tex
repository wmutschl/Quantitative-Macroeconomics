\begin{enumerate}
\item Variables: gross domestic product (GDP), personal income, corporate profits, 
  government spending, tax revenue, government deficit, unemployment rate, consumer price indices,
  interest rates, oil price, Greenhouse Gas emissions, debt, stock price indices, house prices,
  population, banking, debt securities, credit, global liquidity, derivatives, foreign exchange, property prices,
  and so on
\item Here are some databases:
\begin{itemize}
    \item European-centric sources:
        \begin{itemize}
            \item Eurostat: official source for statistical data on the European Union, member states and sub-state regions
            \item ECB Statistical Data Warehouse:
            Data on Euro area monetary policy, financial stability and the activities of the European System of Central Banks (ESCB),
              with aggregate series based on observations from national central banks, credit institutions and international data sources.
            \item Data Europa EU (the former European Data Portal) provides access to over a million public datasets from 36 countries
            (European Union member states, the EEA, Switzerland and countries in the EU Neighborhood Policy programme).
            Data resources are indexed by the European Commission from national, regional, local and domain-specific public data providers. 
        \end{itemize}
    \item US-centric sources: 
        \begin{itemize}
            \item FRED database: User-friendly database of U.S. and \textbf{international} time series data maintained by the Federal Reserve Bank of St. Louis
            \item Bureau of Labor Statistics (BLS): publishes data on unemployment and consumer prices, as well as a host of data related to the U.S. labor force
            \item Bureau of Economic Analysis (BEA): compiles extensive data on gross domestic product (GDP), personal income, and corporate profits
            \item National Bureau of Economic Research (NBER): hosts data covering the U.S. economy, industry, and international trade
            \item Congressional Budget Office: data on federal spending and revenue, projections of future spending and deficits, and forecasts
        \end{itemize}
    \item Worldwide:
        \begin{itemize}
            \item Central or national banks maintain many macroeconomic statistics
            \item World Bank maintains many large datasets across most countries,
              including the World Development Indicators (WDI) database
              and the Global Financial Development Database (GFDD).
            While they maintain an extensive set of data series with expansive country coverage,
              there are some missing data issues in many of the series.
            \item Bank for International Settlements:
              datasets on international banking, debt securities, credit, global liquidity, derivatives, foreign exchange, property prices
            \item United Nations maintains numerous databases, tables, and glossaries
              containing over 60 million data points covering international economic, health, education, and development data
            \item OECD.Stat includes data and metadata for OECD countries and selected non-member economies
        \end{itemize}
    \item Others/Data aggregator (commercial)
        \begin{itemize}
            \item Our World in Data
            \item DB NOMICS
            \item Datastream and EIKON
            \item Macrobond
        \end{itemize}		
	\end{itemize}
\item It all depends on the analysis you want to conduct,
  i.e. whether you need country-, sector- or firm-specific data.
Try to get a name for a category the dataset might belong to.
Tipp: Go through the list of Providers on DBNomics to see which categories there are.
Our World in Data also tends to give good ideas where to get data from.
Always try official sources first,
  i.e. national statistical offices or central banks,
  then large international organizations, i.e. World Bank, OECD or BIS.
Also when reading other papers, have a close look which sources have been used.
\item
The data was downloaded on October 26, 2022 from FRED and saved into a CSV (comma-seperated values) file \texttt{NorwayGDP.csv}.
The shortcode is \texttt{CLVMNACSCAB1GQNO}: Real Gross Domestic Product for Norway, Millions of Chained 2010 National Currency, Quarterly, Seasonally Adjusted.
The time period included is 1978-01-01 to 2022-04-01.
Use MATLAB's \texttt{Import Data} feature and select the columns and rows of the data.
Edit the names of the variables and the input format of the dates;
  it is useful to tell MATLAB that the column for dates are actual dates,
  such that MATLAB creates a so called datetime array,
  and we can easily manipulate dates.
Then click on the arrow under the green checkmark and select \texttt{Generate Script} to get MATLAB code that always will load in your data the same way.
See the following script file:
\lstinputlisting[style=Matlab-editor,basicstyle=\mlttfamily,title=\lstname]{progs/matlab/visualizingTimeSeriesDataNorway.m}
Remarks:
\begin{itemize}
    \item A growth rate can be approximated using logs $$\frac{Y_t - Y_{t-p}}{Y_{t-p}} \approx log \left(\frac{Y_t}{Y_{t-p}}\right)$$
      where $p=1$ would correspond to one period lagged. If we want to compute year-on-year rates we would set $p=4$ for quarterly data.
    \item Normal distribution is not a very good choice, as we have asymmetry and some probability mass in the tails of the distribution.		
\end{itemize}
\item See above for the code. The distributions differ significantly depending on the subsample considered.
The assumption of normality is violated; thus, we are faced with skewed distributions and have to deal with possible outliers.
This is very typical for macroeconomic time series data, see e.g. \textcite{Ascari.Fagiolo.Roventini_2015_FatTailDistributionsBusinessCycle} or \textcite{Fagiolo.Napoletano.Roventini_2008_AreOutputGrowthrate}.
However, we will see that during the course we will still keep the normality assumption;
  so keep this in mind for later whether this is correct or not.
\item Optional:
\begin{itemize}
    \item Clone the repository.
    \item Download data for e.g. Germany and put that into \texttt{data/GermanyGDP.csv}
    \item Create a MATLAB script with your codes under \texttt{progs/matlab/visualizingTimeSeriesDataGermany.m}
    \item Uncomment the line at the end of \texttt{exercises/visualizing\_time\_series\_data\_solution.inc},
    so that your script is included in the compiled PDF.    
    \item Do a pull request on GitHub.
\end{itemize}
%% Uncomment (i.e. remove the %) from the following line for your pull request, change the names accordingly.
%\lstinputlisting[style=Matlab-editor,basicstyle=\mlttfamily,title=\lstname]{progs/matlab/visualizingTimeSeriesDataGermany.m}
\end{enumerate}