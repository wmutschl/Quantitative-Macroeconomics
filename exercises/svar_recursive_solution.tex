\begin{enumerate}
\item[1.] The model is identified recursively with the real price of oil ordered first:
\begin{align*}
y_t = \begin{pmatrix}\Delta rpoil_t \\ \Delta p_t \\ \Delta gdp_t \end{pmatrix}
\end{align*}
such that the real price of oil is \textbf{predetermined} with respect to the U.S. economy.
In other words, only the structural oil price shock (ordered first) has an immediate effect on the real price of oil,
the other two structural shocks affect the real price of oil with a delay and not on impact.
The ordering is thus very important here,
as our focus is on the effect of an \textbf{unanticipated (exogenous)} increase in the real price of oil.
Moreover, as we are only interested in one shock,
the model is only partially identified in that only the oil price shock can be given an economic interpretation.

\item[2./3./4./5.]  \lstinputlisting[style=Matlab-editor,basicstyle=\mlttfamily,title=\lstname]{progs/matlab/USOil.m}
Here is the helper function for the more general approach:
\lstinputlisting[style=Matlab-editor,basicstyle=\mlttfamily,title=\lstname]{progs/matlab/USOil_fSR.m}
\item[6.] In the period under consideration, crude oil is a commodity that is considered vulnerable to negative supply shocks
due to the political and social volatility of its (Middle East) source.
In macroeconomics a negative supply shock is an unexpected event that changes the supply of a product,
resulting in an increase in prices and a decrease in output.
From the IRFs we see exactly this pattern:
an unexpected oil price shock creates inflationary pressure on the GDP deflator and a reduction in real GDP in the US\@.
In this sense, a positive oil price shock (originating e.g.\ from the oil producing countries) indeed acts like a negative domestic supply shock for the U.S. economy.
\end{enumerate}
